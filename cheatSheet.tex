\documentclass[a4paper]{article}

%% Language and font encodings
\usepackage[english]{babel}
\usepackage[utf8x]{inputenc}
\usepackage[T1]{fontenc}

%% Sets page size and margins
\usepackage[a4paper,top=1cm,bottom=1cm,left=1cm,right=1cm,marginparwidth=1.75cm]{geometry}

%% Useful packages
\usepackage{amsmath}
\usepackage{amssymb}
\usepackage{graphicx}
\usepackage[colorinlistoftodos]{todonotes}
\usepackage[colorlinks=true, allcolors=blue]{hyperref}

\begin{document}
	\begin{itemize}
    	\item Linear/undetermined systems
        \begin{itemize}
        	\item Matrix multiplication is \(\rightarrow \downarrow\).
			\item \textbf{Row echelon form} is the result of Gaussian elimination.
        	\item \textbf{Pivots} are first nonzero value in a row. \textbf{Free variables} are those columns without pivots. \textbf{Dependent variables} are columns with pivots.
            \item \textbf{Specific (particular) solutions} are solutions \(x_s\), and \textbf{vectors in the null space} are solutions \(x_n\)
          	\item \(Ax=b\) for multiple solutions is the \textbf{general (total) solution}: \(A(x_s+\beta x_n)=b\)

            where \(x_s\) is s.t. \(Ax_s=b\) and \(x_n\) s.t. \(Ax_n=0\)
            \item Vectors form a basis for a column space of \(A\) if they're in \(A\) and linearly independent.
            \item Vectors in a row space are column vectors.
        \end{itemize}

		\item QR factorization:
    		\begin{itemize}
    			\item \textbf{Normal equation}: \(A^TA\hat{x}=A^Tb\) or \((A^TA)^{-1}A^Tb=\hat{x}\)

                Plug in \(A\) and \(b\) to find \(\hat{x}\) = best approximate solution (\textbf{linear least-squares solution})

				\item \textbf{Compute projection} of \( b \) onto \(A\), call it \(\hat{b}\): \(A(A^TA)^{−1}A^Tb=\hat{b}\).

                Note that this is just the \(\hat{x}\) from the normal equation multiplied by \(A\), so \(\hat{b}=A\hat{x}\).

				\item Orthonormal vectors: $$q_k=\frac{a^\bot _k}{\rho _{k,k}}=\frac{a^{\bot }_k}{\parallel a^{\bot }_k \parallel _2}$$ where
					$$a^\bot _k = a_k − \rho _{0,k}q_0− \dots −\rho _{k−1,k}q_{k−1}$$
					$$a^\bot _k=a_k−q^T_0a_kq_0−\dots −q^T_{k−1}a_kq_{k−1}$$

                \item \(A=QR\) where \(Q=(q_0 | \dots | q_{n-1})\)

                	and \(R=
                		\begin{pmatrix}
                    	\parallel a_0 \parallel _2 & \rho _{0,1} & \dots & \rho _{0,n-1} \\
                        & \parallel a_1^{\bot } \parallel _2 & \ddots & \vdots \\
                        & & \ddots & \rho _{n-2,n-1} \\
                        0 & & & \parallel a^{\bot }_{n-1} \parallel _2
                    	\end{pmatrix}
                	\). Basically any \(\rho _{k,k}=\parallel a_k^{\bot } \parallel _2 \),
                	\(\rho _{(i<k),k}=q_i^Ta_k\), and \(\rho _{(i>k),k}=0\).
                \item \(Q^TQ=I\)
			\end{itemize}
		\item Space spanned by vectors: \(A=\{ a_0|\text{...}|a_{n−1}\} \) where A is the space
        \item \textbf{Eigenvalues} are scalars \(\lambda \). \(\lambda\) is an eigenvalue of \(A \iff Ax = \lambda x\) for some non-zero vector \(x\). So:

        	For 2x2 \(M =
                \begin{pmatrix}
                    a-\lambda & b \\
                    c & d-\lambda
                \end{pmatrix}\), and for 3x3 \( M =
                \begin{pmatrix}
                    a-\lambda & b & c \\
                    d & e-\lambda & f \\
                    g & h & i-\lambda
                \end{pmatrix} \),

            and \( \text{det}(A-\lambda I)=0\).
        \item \textbf{Eigenvectors}
        \item Determinants:
        	\begin{itemize}
        		\item For 2x2 matrices
                	\(M =
                    \begin{pmatrix}
                		a & b \\
                        c & d
                	\end{pmatrix}\),
                    \(\text{det}(M)=ad-bc \)
                \item For 3x3 matrices
                	\(M =
                    \begin{pmatrix}
                		a & b & c \\
                        d & e & f \\
                        g & h & i
                	\end{pmatrix}\),
                    \(\text{det}(M)= a(ei-hf)-b(di-fg)+c(dh-eg) \), or

                    \(\text{det}(M)= aei+bfg+cdh-(afh+bdi+ceg) \)
        	\end{itemize}
        \item Equivalent to "A is nonsingular"
        	\begin{itemize}
        		\item \(A\) is invertible.
                \item \(A^{-1}\) exists.
                \item \(AA^{-1}=A^{-1}A=I\).
                \item \(A\) represents a linear transformation that is a bijection.
                \item \(Ax=b\) has a unique solution for all \(b\in \mathbb{R} ^n\).
                \item \(Ax=0\) implies that x=0.
                \item \(Ax=e_j\) has a solution for all \(j\in \{0,\dots ,n-1\}\)
                \item The determinant of \(A\) is nonzero: \(det(A)\neq 0\).
                \item \(LU\) with partial pivoting does not break down.
                \item \(\mathcal{N}(A)={0}\).
                \item \(\mathcal{C}(A)=\mathbb{R}^n\).
                \item \(\mathcal{R}(A)=\mathbb{R}^n\).
                \item \(A\) has linearly independent columns.
                \item \(A\) has linearly independent rows.
        	\end{itemize}
	\end{itemize}
\end{document}
